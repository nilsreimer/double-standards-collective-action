\documentclass[12pt, letterpaper]{article}

% Font
\usepackage[utf8]{inputenc}
\usepackage{MinionPro} 
\input{glyphtounicode}
\pdfgentounicode=1 
\usepackage{microtype}

% Format
\usepackage[letterpaper, margin = 1in]{geometry}
\setcounter{secnumdepth}{0}
\usepackage{titlesec}  
\titleformat*{\section}{\centering\normalfont\Large\bfseries}

% Links
\usepackage[colorlinks = true, linkcolor = black, urlcolor = black, citecolor = black]{hyperref} 

% Figures
\usepackage{graphicx}
\usepackage[labelfont = bf, font = small, labelsep = newline, singlelinecheck = false]{caption}
\renewcommand{\thefigure}{S\arabic{figure}}
\renewcommand{\thetable}{S\arabic{table}}

% Tables
\usepackage{booktabs}
\usepackage{xltabular}

% References
\newenvironment{CSLReferences}[2]{}{}

% Frontmatter
\title{ Supplemental Online Materials:\\\textit{Double Standards in
Judging Collective Action} }
\author{  }

\begin{document}

\maketitle

\hypertarget{scale-development}{%
\section{Scale Development}\label{scale-development}}

\hypertarget{study-1}{%
\subsection{Study 1}\label{study-1}}

In Study 1, we compiled a list of collective actions from participants'
responses and other sources. To that end, we recruited 60 participants
from the Prolific subject pool, all of whom were citizens of the UK or
the US. To increase the socioeconomic diversity of our sample, we
recruited 30 non-students without a university degree, 15 non-students
with a university degree, and 15 current university students.
Participants first read an accessible description of collective action:

\begin{quote}
Society is not only made up of individuals, but consists of `social
groups' to which these individuals belong. Each person lives in a place,
has a job (or not) at an organisation, is a fan of a specific sports
club, has a religion, or belongs to any number of other such groups.
Individuals often act in ways to promote the interests of the social
groups to which they belong.
\end{quote}

\begin{quote}
For example, the workers in a factory might want to get paid more. They
might go on a strike to reach that goal. Students at a university might
want to prevent an increase in tuition fees. They might hand out flyers
or occupy a building to reach that goal. Other groups might want to
protest against police violence. They might block traffic on a road to
make others aware of this issue.
\end{quote}

\noindent Participants were then asked to name at least five (and up to
ten) actions that fit that description and were encouraged to think of
actions that vary in how acceptable or unacceptable they are (in their
opinion):

\begin{quote}
Please think of other actions that members of a social group (or social
groups as a whole) might take to promote their group's interests or
goals. Please name at least 5 actions that fit the description above.
You can name up to 10 actions if you want to. Please try to think of
actions that vary in how acceptable or unacceptable they are (in your
opinion).
\end{quote}

\noindent After naming at least five actions, participants sorted their
responses into actions that, across a range of situations, were ``always
acceptable'', ``sometimes acceptable'', and ``never acceptable'':

\begin{quote}
On the left, you see the actions you have named on the previous page.
Think about each action. Can you think of situations in which this
action is an acceptable means to advance a group's goals or interests?
Can you think of situations in which this action is not an acceptable
means to advance a group's interests?
\end{quote}

\noindent Participants named between 5 and 10 actions
(\(\textit{Mdn} = 7\)). Participants named more actions that they
considered always acceptable (\(\textit{Mdn} = 3\)) than actions they
considered sometimes (\(\textit{Mdn} = 2\)) or never
(\(\textit{Mdn} = 1\)) acceptable. We recoded participants' responses
into a smaller set of unique collective actions. We then supplemented
the resulting list of actions with collective actions from the
psychological and political science literature (e.g., Sharp, 1973). This
process resulted in 72 actions that we expected to vary in how
acceptable most people would find them to be.

\hypertarget{study-2}{%
\subsection{Study 2}\label{study-2}}

In Study 2, we measured how acceptable participants judged the actions
from Study 1 to be and applied item response theory to develop an
instrument to capture double standards in judging collective action. We
recruited 158 participants (\(\textit{Mdn} = 30\) years, age range:
18--68 years; 103 women, 52 men, 2 other, 1 prefer not to say) from the
Prolific subject pool, all of whom were citizens of the UK or the US. To
increase the socioeconomic diversity of our sample, we recruited 80
non-students without a university degree, 37 non-students with a
university degree, and 41 current university students. We excluded 15
participants who failed an attention check, leaving a final sample of
143 participants for our analyses.

Participants first read an accessible description of collective action:

\begin{quote}
Society is not only made up of individuals, but consists of `social
groups' to which these individuals belong. Each person lives in a place,
has a job (or not) at a firm or organisation, has a religion (or not),
is part of a political party, or belongs to any number of other groups
like these.
\end{quote}

\begin{quote}
Individuals often act in ways to promote their group's interest. For
example, individuals might want people like them to get paid more.
Members of a group might take various kinds of actions (for example, go
on a strike).
\end{quote}

\noindent Participants were then asked to ``suppose that one or more
members of a group took the following action to advance a cause in their
group's interest'' and were presented with one of the actions from Study
1. Participants answered several questions about the protest action.
First, they rated how disruptive, violent, and extreme they considered
this action to be (1 = \emph{not at all}, 4 = \emph{very}). Second, they
were asked to think of a range of different causes and circumstances,
and to rate how often this action would be an acceptable means for a
group to advance one of these causes (1 = \emph{never}, 2 =
\emph{rarely}, 3 = \emph{sometimes}, 4 = \emph{often}, 5 =
\emph{always}). Finally, they rated how positive or negative they felt,
in general, about this action (1 = \emph{very positive}, 5 = \emph{very
negative}). Each participant answered these questions for 20 of the 72
actions from Study 1 so that each action was rated by 29--53
participants.

We estimated a graded response model (Bürkner, 2021; Samejima, 1997), an
item response theory model for ordinal response variables, for
participants' ratings of how often an action would be an acceptable form
of collective action. For each participant \(j\), the model estimated
their unique propensity (\(\theta_j\)) to consider collective actions
acceptable in more situations. For each item \(i\), the model estimated
four acceptability thresholds (\(\beta_{ik}\)), separating the five
response options, and one discrimination parameter (\(\alpha_i\))
indicating how well the item differentiates between participants with
different propensities to consider collective actions acceptable. We
estimated the model in \emph{CmdStanR} using similar prior distribution
as in Experiment. We used an induced Dirichlet prior (Betancourt, 2019)
for the item-specific difficulty thresholds.

Table S1 shows each actions' item information and discrimination as well
as proportions for the observed responses. Table S2 shows participants'
averaged ratings of how acceptable, disruptive, violent, extreme, and
negative they considered each action to be. Table S3 shows that all five
ratings were strongy correlated across actions and participants. We
considered both estimates from the graded response model and their
relevance to the social contexts to select actions for Experiments 1 and
2.

\hypertarget{experiment-1}{%
\section{Experiment 1}\label{experiment-1}}

\hypertarget{method}{%
\subsection{Method}\label{method}}

\hypertarget{procedure}{%
\subsubsection{Procedure}\label{procedure}}

We used a two-step process to recruit participants who satisfied our
preregistered inclusion criteria.

First, we recruited a larger pool of 875 participants. Of these, 475
participants did not have a university degree and placed themselves on
the bottom three ranks of the subjective socio-economic status ladder
(\emph{lower-status group}) and 400 participants had at least an
undergraduate degree and placed themselves on the top four ranks of the
subjective socio-economic status ladder (\emph{higher-status group}).
Participants completed a screening survey in which they, among other
questions, answered whether they considered either their current job or
the jobs they had had in the past or would have in the future to be a
working-class job or a middle-class/professional job (1 =
\emph{working-class job}, 2 = \emph{middle-class/professional job}, 3 =
\emph{neither}). Before answering this question, participants read the
following explanation:

\begin{quote}
The workforce is often divided into two kinds of jobs. What we call
working-class jobs are jobs done by skilled, semi-skilled, unskilled
manual workers or by casual workers. These jobs do not usually require a
university degree. What we call middle-class or professional jobs are
administrative, managerial, or other jobs that usually require a
university degree.
\end{quote}

\noindent As preregistered, we excluded participants from the
lower-status group who had responded ``middle-class/professional job''
or ``neither'' to this question; we excluded participants from the
higher-status group who had responded ``working-class job'' or
``neither'' to this question. This left 687 participants for the next
step of the selection procedure.

Second, we set out to recruit 500 participants from the remaining 687
participants, 250 from the lower-status group and 250 from the
higher-status group. After giving their informed consent, participants
were randomly assigned to read a vignette about a government bill
affecting either people in working-class jobs (\emph{lower-status
protesters}) or people in professional jobs (\emph{higher-status
protesters}). Participants in both conditions were instructed to
carefully read the vignette and to try to imagine what it would be like
if this situation was real. Participants in the lower-status protesters
condition then read the following introduction:

\begin{quote}
The government, though not necessarily the current government, is going
to introduce a bill that will mostly affect people in working-class
jobs. Working-class jobs, in this case, are jobs done by skilled,
semi-skilled, unskilled manual workers or by casual workers. These are
jobs that do not usually require a university degree. Other jobs are
unlikely to be affected.
\end{quote}

\noindent Participants in the higher-status protesters condition instead
read the following introduction:

\begin{quote}
The government, though not necessarily the current government, is going
to introduce a bill that will mostly affect people in professional jobs.
Professional jobs are administrative, managerial, or other jobs that
usually require a university degree. Other jobs are unlikely to be
affected.
\end{quote}

\noindent Participants in the two conditions then read this almost
identically worded paragraph:

\begin{quote}
This government measure would make it easier for companies to hire
workers during economic growth and to lay off workers during an economic
crisis. As a consequence, companies would be able to fire employees with
little notice and without giving a reason. Trade unions are opposed to
the measure. They argue that the bill would compromise job security, and
prevent employees from challenging harassment or other abuse without the
fear of being fired. People in {[}working-class/professional{]} jobs are
particularly at risk, and there is a rise in tension and outrage among
them.
\end{quote}

After reading the vignette, participants responded to the other measures
(in the order in which they are presented in the main text). On the
final page, we reminded participants that they had read about a bill
that the government planned to introduce and that some people objected
to. As an attention check, we then asked participants to recall who the
people most affected by this measure were. As preregistered, we excluded
all participants who did not respond with an accurate description of the
group they had read about.

\hypertarget{experiment-3}{%
\section{Experiment 3}\label{experiment-3}}

\hypertarget{method-1}{%
\subsection{Method}\label{method-1}}

\hypertarget{measures}{%
\subsubsection{Measures}\label{measures}}

We included additional measures not used in the analyses reported in
this paper.

We included a three-item measure of moral conviction (Ryan, 2014) that
assessed to what extent participants' feelings about restricting or
banning abortion were, for example, connected to their core moral
beliefs or convictions (1 = \emph{not at all}, 5 = \emph{very much}).

We measured gender-related system-justifying beliefs with eight items
(adapted from Jost \& Kay, 2005), for example, ``in general, relations
between men and women are fair'' (1 = \emph{strongly disagree}, 7 =
\emph{strongly agree}; McDonald's \(\omega = .93\)). A confirmatory
factor analysis model in which all items loaded onto a single factor
showed acceptable fit,
\(\chi^2 (20) = 111.90; \text{CFI} = 0.97; \text{TLI} = 0.95; \text{RMSEA} = 0.09, [0.08, 0.11]\).

We measured moral concerns with the 36-item moral foundations
questionnaire (MFQ-2, Atari et al., 2022) which assesses to what extent
participants endorse concerns about care, equality, proportionality,
loyalty, authority, and purity (e.g., ``I believe chastity is an
important virtue.''; 1 = \emph{does not describe me at all}, 5 =
\emph{describes me extremely well}). We embedded three further attention
checks within the questionnaire (e.g., ``To show that you are paying
attention and giving your best effort, please select `moderately
describes me'.'').

\hypertarget{reanalysis-of-teixeira-et-al.s--teixeira_white_2022-experiment}{%
\section{Reanalysis of Teixeira et al.'s (2022)
experiment}\label{reanalysis-of-teixeira-et-al.s--teixeira_white_2022-experiment}}

At the height of the 2020 Black Lives Matter protests, Teixeira et al.
(2022) exposed White Americans (\(N = 399\)) to photos of either
peaceful or destructive Black Lives Matter protests and asked them to
rate how legitimate (i.e., fair, reasonable, legitimate, and justified)
they thought the pictured protests were. As such, their findings make
for an interesting comparison to our findings. Teixeira et al. (2022),
however, reported results from a linear model estimating the three-way
interaction between the experimental manipulation, belief in systemic
racial injustice, and group identification while controlling for
political orientation. This analysis did not provide a direct comparison
to our findings.

To compare theirs to our findings, we instead estimated a series of
simpler linear regression models, using the brms R package (Bürkner,
2017, 2018) and weakly informative Student-\emph{t} \((3, 0, 2.5)\)
prior distributions, with participants' ratings of how legitimate the
pictured protests were as the outcome variable.\footnote{We thank the
  authors for making their data available online
  (\url{https://osf.io/sud2y/}).} First, we found that, as expected,
participants considered peaceful protests far more legitimate than
destructive protests (\(\text{Cohen's}~d = 1.14, [0.98, 1.30]\)).
Second, we found that more conservative participants rated destructive
protests (\(\beta_{xy} = -0.59, [-0.68, -0.49]\)) and, to a lesser
extent, peaceful protests (\(\beta_{xy} = -0.20, [-0.30, -0.10]\)) to be
less legitimate than more liberal participants. Third, we found that
participants who identified more strongly with their (White American)
ethnic identity rated destructive protests
(\(\beta_{xy} = -0.35, [-0.45, -0.24]\)) and, to a lesser extent,
peaceful protests (\(\beta_{xy} = -0.10, [-0.21, 0.01]\)) to be less
legitimate than participants who identified less strongly with their
(White American) ethnic identity. Finally, we found that participants'
group identification was no longer associated with how legitimate they
rated peaceful (\(\beta_{xy} = -0.02, [-0.13, 0.05]\)) and destructive
(\(\beta_{xy} = -0.06, [-0.17, 0.05]\)) protests to be after controlling
for their political orientation. We refer to findings from our
reanalysis of Teixeira et al.'s (2022) experiment in the main text.

\newpage

\hypertarget{references}{%
\section{References}\label{references}}

\begingroup

\noindent \setlength{\parindent}{-0.5in} \setlength{\leftskip}{0.5in}

\hypertarget{refs}{}
\begin{CSLReferences}{1}{0}
\leavevmode\vadjust pre{\hypertarget{ref-atari_morality_2022}{}}%
Atari, M., Haidt, J., Graham, J., Koleva, S., Stevens, S. T., \&
Dehghani, M. (2022). Morality beyond the {WEIRD}: How the nomological
network of morality varies across cultures. \emph{PsyArXiv}.
\url{https://doi.org/10.31234/osf.io/q6c9r}

\leavevmode\vadjust pre{\hypertarget{ref-betancourt_ordinal_2019}{}}%
Betancourt, M. (2019). \emph{Ordinal regression}.
\url{https://betanalpha.github.io/assets/case_studies/ordinal_regression.html}

\leavevmode\vadjust pre{\hypertarget{ref-burkner_brms_2017}{}}%
Bürkner, P.-C. (2017). {brms}: An {R} package for bayesian multilevel
models using {Stan}. \emph{Journal of Statistical Software},
\emph{80}(1), 1--28. \url{https://doi.org/10.18637/jss.v080.i01}

\leavevmode\vadjust pre{\hypertarget{ref-burkner_advanced_2018}{}}%
Bürkner, P.-C. (2018). Advanced bayesian multilevel modeling with the
{R} package brms. \emph{The R Journal}, \emph{10}(1), 395--411.
\url{https://doi.org/10.32614/RJ-2018-017}

\leavevmode\vadjust pre{\hypertarget{ref-burkner_bayesian_2021}{}}%
Bürkner, P.-C. (2021). Bayesian item response modeling in {R} with brms
and {Stan}. \emph{Journal of Statistical Software}, \emph{100}(5).
\url{https://doi.org/10.18637/jss.v100.i05}

\leavevmode\vadjust pre{\hypertarget{ref-jost_exposure_2005}{}}%
Jost, J. T., \& Kay, A. C. (2005). Exposure to benevolent sexism and
complementary gender stereotypes: Consequences for specific and diffuse
forms of system justification. \emph{Journal of Personality and Social
Psychology}, \emph{88}(3), 498--509.
\url{https://doi.org/10.1037/0022-3514.88.3.498}

\leavevmode\vadjust pre{\hypertarget{ref-ryan_reconsidering_2014}{}}%
Ryan, T. J. (2014). Reconsidering moral issues in politics. \emph{The
Journal of Politics}, \emph{76}(2), 380--397.
\url{https://doi.org/10.1017/S0022381613001357}

\leavevmode\vadjust pre{\hypertarget{ref-samejima_graded_1997}{}}%
Samejima, F. (1997). Graded response model. In W. J. van der Linden \&
R. K. Hambleton (Eds.), \emph{Handbook of modern item response theory}
(pp. 85--100). Springer.

\leavevmode\vadjust pre{\hypertarget{ref-sharp_nonviolent_1973}{}}%
Sharp, G. (1973). \emph{The politics of nonviolent action}. Porter
Sargent.

\leavevmode\vadjust pre{\hypertarget{ref-teixeira_white_2022}{}}%
Teixeira, C. P., Leach, C. W., \& Spears, R. (2022). White {Americans}'
belief in systemic racial injustice and in-group identification affect
reactions to ({peaceful} vs. {destructive}) {``{Black Lives Matter}''}
protest. \emph{Psychology of Violence}, \emph{12}(4), 280--292.
\url{https://doi.org/10.1037/vio0000425}

\end{CSLReferences}

\endgroup

\newpage

\begin{xltabular}{\linewidth}{rXrrrrrrr}

\caption{Results from Study 2}\\
\toprule
   &        &     &          & \multicolumn{5}{c}{Response $[\%]$} \\ \cmidrule{5-9}
\# & Action & $I$ & $\alpha$ & 1 & 2 & 3 & 4 & 5\\
\midrule
\endfirsthead

\toprule
   &        &     &          & \multicolumn{5}{c}{Response $[\%]$} \\ \cmidrule{5-9}
\# & Action & $I$ & $\alpha$ & 1 & 2 & 3 & 4 & 5\\
\midrule
\endhead

\bottomrule
\addlinespace
\caption*{\textit{Note.} $I$ = Information; $\alpha$ = Discrimination; Response: 1 = \textit{never}, 2 = \textit{rarely}, 3 = \textit{sometimes}, 4 = \textit{often}, 5 = \textit{always}}
\endfoot

\bottomrule
\addlinespace
\caption*{\textit{Note.} $I$ = Information; $\alpha$ = Discrimination; Response: 1 = \textit{never}, 2 = \textit{rarely}, 3 = \textit{sometimes}, 4 = \textit{often}, 5 = \textit{always}}
\endlastfoot

1 & disrupt traffic (e.g., blocking roads) & 7.98 & 1.31 & 41 & 36 & 18 & 5 & 0\\
2 & attend or organise a protest rally & 7.35 & 1.11 & 0 & 7 & 45 & 39 & 9\\
3 & refuse to work (strike) & 6.96 & 1.11 & 2 & 5 & 53 & 33 & 7\\
4 & enter and refuse to leave a building (occupation) & 6.95 & 1.12 & 6 & 52 & 26 & 13 & 3\\
5 & deface flags or other national symbols & 6.83 & 1.21 & 52 & 20 & 20 & 9 & 0\\

6 & refuse to honour national symbols and traditions (e.g., refusing to sing the national anthem) & 6.42 & 1.19 & 15 & 18 & 40 & 15 & 12\\
7 & paste up posters with political messages in places where it is not allowed or encouraged & 6.37 & 1.01 & 14 & 32 & 35 & 19 & 0\\
8 & refuse to cooperate with the police and other government agencies & 6.37 & 1.11 & 22 & 38 & 32 & 5 & 3\\
9 & refuse to accept honours or awards in protest & 6.32 & 1.14 & 11 & 14 & 39 & 23 & 14\\
10 & disrupt public events (e.g., a sports game) with a political message & 6.30 & 1.02 & 24 & 45 & 21 & 10 & 0\\

11 & spray paint political messages in public places & 6.28 & 1.06 & 31 & 39 & 28 & 3 & 0\\
12 & stand or sit in a building and refuse to leave (stand-in, sit-in) & 6.28 & 1.04 & 7 & 41 & 34 & 14 & 5\\
13 & pay for adverts on social media (e.g., Facebook, Twitter, Instagram) to influence public opinion & 6.23 & 0.99 & 0 & 16 & 27 & 39 & 18\\
14 & get involved in the media (e.g., newspapers, radio, television) to influence the public & 6.18 & 0.98 & 0 & 11 & 32 & 36 & 20\\
15 & visit people in their homes to convince them about an issue (canvassing, door knocking) & 6.07 & 1.03 & 9 & 30 & 27 & 27 & 6\\

16 & donate to political parties who support the cause & 6.00 & 1.10 & 6 & 6 & 33 & 31 & 25\\
17 & do not buy goods or services from companies who oppose the cause (consumers' boycott) & 5.99 & 1.06 & 2 & 5 & 39 & 29 & 24\\
18 & refuse to interact with or acknowledge individuals who oppose the cause & 5.89 & 0.97 & 21 & 45 & 29 & 5 & 0\\
19 & attend or organise a protest march & 5.88 & 1.06 & 3 & 3 & 34 & 38 & 22\\
20 & paste up posters with political messages in places where it is allowed and encouraged & 5.85 & 1.00 & 0 & 3 & 20 & 43 & 34\\

21 & refuse to pay rent (rent strike) & 5.82 & 0.97 & 23 & 41 & 28 & 8 & 0\\
22 & use social media (e.g., Facebook, Twitter, Instagram) to influence the public & 5.82 & 1.01 & 5 & 5 & 30 & 45 & 15\\
23 & wear or display political symbols & 5.79 & 1.00 & 3 & 9 & 38 & 32 & 18\\
24 & mock or insult individuals who oppose the cause & 5.73 & 1.11 & 53 & 34 & 3 & 11 & 0\\
25 & write letters to politicians, representatives and elected officials & 5.69 & 1.07 & 0 & 0 & 11 & 37 & 53\\

26 & use social media (e.g., Facebook, Twitter, Instagram) to inform the public & 5.67 & 1.01 & 0 & 3 & 12 & 41 & 44\\
27 & join or form a group of activists & 5.66 & 1.02 & 2 & 12 & 21 & 38 & 26\\
28 & hold meetings to inform the public & 5.65 & 1.01 & 0 & 3 & 11 & 38 & 49\\
29 & donate to activist groups who support the cause & 5.64 & 1.06 & 2 & 9 & 41 & 14 & 34\\
30 & disrupt public services (e.g., shutting down government websites) & 5.61 & 1.01 & 42 & 42 & 17 & 0 & 0\\

31 & refuse to pay fees, fines, and taxes & 5.56 & 0.96 & 9 & 50 & 30 & 7 & 4\\
32 & hold meetings to influence the public & 5.56 & 1.00 & 0 & 3 & 16 & 35 & 46\\
33 & hand out flyers, leaflets, or pamphlets & 5.55 & 1.01 & 0 & 0 & 13 & 45 & 42\\
34 & visit people in their homes to inform them about an issue (canvassing, door knocking) & 5.39 & 0.95 & 9 & 20 & 30 & 32 & 9\\
35 & make a public speech & 5.35 & 1.00 & 2 & 2 & 14 & 46 & 36\\

36 & join or form a political party & 5.28 & 0.96 & 2 & 7 & 19 & 45 & 26\\
37 & refuse to interact with or acknowledge politicians who oppose the cause & 5.27 & 0.95 & 8 & 35 & 38 & 10 & 10\\
38 & disrupt private life of politicians (e.g., protesting outside their home) & 5.24 & 0.92 & 39 & 36 & 23 & 2 & 0\\
39 & get involved in the media (e.g., newspapers, radio, television) to inform the public & 5.23 & 0.93 & 0 & 2 & 11 & 43 & 43\\
40 & pay for adverts in the media (e.g., newspapers, radio, television) to influence public opinion & 5.23 & 0.97 & 5 & 10 & 22 & 42 & 20\\

41 & damage private property (e.g., cars or houses) & 5.17 & 1.03 & 70 & 25 & 5 & 0 & 0\\
42 & refuse service (e.g., in a restaurant or shop) to politicians who oppose the cause & 5.16 & 0.89 & 29 & 37 & 26 & 8 & 0\\
43 & sign or start a petition & 5.10 & 0.96 & 0 & 0 & 8 & 39 & 53\\
44 & mock or insult politicians who oppose the cause & 5.07 & 0.99 & 38 & 31 & 16 & 11 & 4\\
45 & join or form a trade/labor union & 5.07 & 0.99 & 2 & 2 & 14 & 37 & 44\\

46 & donate to charities who support the cause & 5.03 & 0.96 & 0 & 0 & 8 & 39 & 53\\
47 & refuse to eat (hunger strike) & 4.95 & 0.97 & 38 & 24 & 29 & 6 & 3\\
48 & damage public property (e.g., goverment buildings) & 4.83 & 0.98 & 72 & 20 & 5 & 2 & 0\\
49 & spread rumours about politicians who oppose the cause & 4.77 & 1.00 & 69 & 21 & 10 & 0 & 0\\
50 & participate in a public meeting of representatives and elected officials & 4.70 & 1.00 & 3 & 0 & 7 & 47 & 43\\

51 & physically harm oneself (e.g., setting oneself on fire) & 4.70 & 1.03 & 80 & 20 & 0 & 0 & 0\\
52 & boycott an election by not voting or spoiling one's ballot & 4.70 & 0.91 & 14 & 38 & 26 & 12 & 10\\
53 & stand in an election & 4.69 & 0.90 & 2 & 8 & 18 & 35 & 38\\
54 & soil politicians who oppose the cause (e.g., throwing eggs at them) & 4.62 & 1.02 & 70 & 21 & 8 & 0 & 2\\
55 & attack politicians with the intention of harming them (e.g., punching them) & 4.62 & 1.01 & 78 & 22 & 0 & 0 & 0\\

56 & refuse service (e.g., in a restaurant or shop) to individuals who oppose the cause & 4.60 & 0.91 & 32 & 54 & 11 & 0 & 3\\
57 & blackmail individuals who oppose the cause & 4.53 & 0.96 & 75 & 22 & 3 & 0 & 0\\
58 & threaten politicians who oppose the cause with physical harm & 4.37 & 1.00 & 86 & 11 & 3 & 0 & 0\\
59 & attack individuals who oppose the cause with the intention of harming them (e.g., punching them) & 4.36 & 0.95 & 82 & 12 & 6 & 0 & 0\\
60 & damage commerical property (e.g., shop windows) & 4.36 & 1.03 & 88 & 12 & 0 & 0 & 0\\

61 & attack politicians with the intention of killing them (e.g., stabbing them) & 4.30 & 1.01 & 91 & 7 & 2 & 0 & 0\\
62 & bribe politicians, representatives, and other elected officials & 4.24 & 0.95 & 85 & 13 & 2 & 0 & 0\\
63 & voting for candidates/parties & 4.23 & 0.85 & 0 & 5 & 7 & 21 & 67\\
64 & donate to political parties to make them change their position on an issue & 4.21 & 0.90 & 35 & 24 & 30 & 3 & 8\\
65 & blackmail politicians who oppose the cause & 4.20 & 0.88 & 72 & 26 & 3 & 0 & 0\\

66 & attack police officers and other government agents with the intention of killing them (e.g., stabbing them) & 4.20 & 1.02 & 93 & 7 & 0 & 0 & 0\\
67 & attack police officers and other government agents with the intention of harming them (e.g., punching them) & 4.17 & 1.00 & 92 & 5 & 3 & 0 & 0\\
68 & attack individuals who oppose the cause with the intention of killing them (e.g., stabbing them) & 3.98 & 1.00 & 96 & 2 & 2 & 0 & 0\\
69 & threaten individuals who oppose the cause with physical harm & 3.96 & 0.99 & 94 & 6 & 0 & 0 & 0\\
70 & attack members of the public (e.g., by setting off a bomb in a public place) & 3.72 & 0.99 & 98 & 0 & 2 & 0 & 0\\

71 & spread misinformation to influence public opinion & 3.71 & 0.82 & 78 & 19 & 3 & 0 & 0\\
72 & threaten to attack members of the public (e.g., by making a bomb threat) & 3.51 & 0.97 & 98 & 0 & 0 & 2 & 0

\end{xltabular}

\newpage

\setcounter{table}{1}

\begin{xltabular}{\linewidth}{rXrrrrr}

\caption{Ratings from Study 2}\\
\toprule
   &        & \multicolumn{5}{c}{Rating $[M]$} \\ \cmidrule{3-7}
\# & Action & Acce. & Disr. & Viol. & Extr. & Nega.\\
\midrule
\endfirsthead

\toprule
   &        & \multicolumn{5}{c}{Rating $[M]$} \\ \cmidrule{3-7}
\# & Action & Acce. & Disr. & Viol. & Extr. & Nega.\\
\midrule
\endhead

\bottomrule
\addlinespace
\caption*{\textit{Note.} Acce. = Acceptable (1 = \textit{never}, 5 = \textit{always}); Disr. = Disruptive, Viol. = Violent, Extr. = Extreme (1 = \textit{not at all}, 4 = \textit{very}); Nega. = Negative (1 = \textit{very positive}, 5 = \textit{very negative})}
\endfoot

\bottomrule
\addlinespace
\caption*{\textit{Note.} Acce. = Acceptable (1 = \textit{never}, 5 = \textit{always}); Disr. = Disruptive, Viol. = Violent, Extr. = Extreme (1 = \textit{not at all}, 4 = \textit{very}); Nega. = Negative (1 = \textit{very positive}, 5 = \textit{very negative})}
\endlastfoot

1 & disrupt traffic (e.g., blocking roads) & 1.86 & 3.82 & 2.05 & 2.98 & 4.16\\
2 & attend or organise a protest rally & 3.50 & 2.50 & 1.36 & 1.48 & 2.64\\
3 & refuse to work (strike) & 3.37 & 3.35 & 1.09 & 2.07 & 2.60\\
4 & enter and refuse to leave a building (occupation) & 2.55 & 3.32 & 1.48 & 2.42 & 3.19\\
5 & deface flags or other national symbols & 1.85 & 2.98 & 2.30 & 2.98 & 4.15\\

6 & refuse to honour national symbols and traditions (e.g., refusing to sing the national anthem) & 2.92 & 1.85 & 1.23 & 1.62 & 3.20\\
7 & paste up posters with political messages in places where it is not allowed or encouraged & 2.59 & 2.84 & 1.22 & 1.92 & 3.38\\
8 & refuse to cooperate with the police and other government agencies & 2.30 & 3.14 & 2.11 & 2.78 & 3.76\\
9 & refuse to accept honours or awards in protest & 3.14 & 1.84 & 1.07 & 1.59 & 2.86\\
10 & disrupt public events (e.g., a sports game) with a political message & 2.17 & 3.40 & 1.60 & 2.31 & 3.79\\

11 & spray paint political messages in public places & 2.03 & 2.72 & 1.53 & 2.44 & 3.97\\
12 & stand or sit in a building and refuse to leave (stand-in, sit-in) & 2.68 & 3.34 & 1.27 & 2.16 & 3.02\\
13 & pay for adverts on social media (e.g., Facebook, Twitter, Instagram) to influence public opinion & 3.59 & 1.59 & 1.02 & 1.30 & 2.75\\
14 & get involved in the media (e.g., newspapers, radio, television) to influence the public & 3.66 & 1.61 & 1.02 & 1.23 & 2.41\\
15 & visit people in their homes to convince them about an issue (canvassing, door knocking) & 2.91 & 2.76 & 1.06 & 1.36 & 3.33\\

16 & donate to political parties who support the cause & 3.64 & 1.14 & 1.00 & 1.17 & 2.47\\
17 & do not buy goods or services from companies who oppose the cause (consumers' boycott) & 3.68 & 2.07 & 1.05 & 1.29 & 2.27\\
18 & refuse to interact with or acknowledge individuals who oppose the cause & 2.18 & 2.42 & 1.24 & 2.37 & 3.79\\
19 & attend or organise a protest march & 3.72 & 2.56 & 1.31 & 1.56 & 2.41\\
20 & paste up posters with political messages in places where it is allowed and encouraged & 4.09 & 1.31 & 1.00 & 1.06 & 2.17\\

21 & refuse to pay rent (rent strike) & 2.21 & 2.72 & 1.10 & 2.41 & 3.67\\
22 & use social media (e.g., Facebook, Twitter, Instagram) to influence the public & 3.60 & 1.40 & 1.00 & 1.15 & 2.42\\
23 & wear or display political symbols & 3.53 & 1.29 & 1.03 & 1.09 & 2.71\\
24 & mock or insult individuals who oppose the cause & 1.71 & 2.79 & 1.95 & 2.58 & 4.18\\
25 & write letters to politicians, representatives and elected officials & 4.42 & 1.39 & 1.05 & 1.03 & 1.53\\

26 & use social media (e.g., Facebook, Twitter, Instagram) to inform the public & 4.25 & 1.31 & 1.00 & 1.06 & 2.03\\
27 & join or form a group of activists & 3.74 & 1.81 & 1.38 & 1.50 & 2.38\\
28 & hold meetings to inform the public & 4.32 & 1.19 & 1.00 & 1.03 & 1.81\\
29 & donate to activist groups who support the cause & 3.68 & 1.39 & 1.11 & 1.25 & 2.39\\
30 & disrupt public services (e.g., shutting down government websites) & 1.75 & 3.75 & 1.61 & 3.06 & 4.28\\

31 & refuse to pay fees, fines, and taxes & 2.48 & 2.61 & 1.04 & 2.09 & 3.46\\
32 & hold meetings to influence the public & 4.24 & 1.14 & 1.00 & 1.03 & 1.86\\
33 & hand out flyers, leaflets, or pamphlets & 4.29 & 1.47 & 1.03 & 1.05 & 2.00\\
34 & visit people in their homes to inform them about an issue (canvassing, door knocking) & 3.11 & 2.45 & 1.07 & 1.50 & 3.16\\
35 & make a public speech & 4.12 & 1.46 & 1.00 & 1.06 & 1.92\\

36 & join or form a political party & 3.86 & 1.40 & 1.00 & 1.24 & 2.10\\
37 & refuse to interact with or acknowledge politicians who oppose the cause & 2.80 & 1.88 & 1.15 & 1.88 & 3.52\\
38 & disrupt private life of politicians (e.g., protesting outside their home) & 1.89 & 3.36 & 2.07 & 2.77 & 4.07\\
39 & get involved in the media (e.g., newspapers, radio, television) to inform the public & 4.27 & 1.36 & 1.02 & 1.14 & 1.82\\
40 & pay for adverts in the media (e.g., newspapers, radio, television) to influence public opinion & 3.62 & 1.60 & 1.10 & 1.35 & 2.50\\

41 & damage private property (e.g., cars or houses) & 1.34 & 3.77 & 3.64 & 3.86 & 4.75\\
42 & refuse service (e.g., in a restaurant or shop) to politicians who oppose the cause & 2.13 & 2.95 & 1.26 & 2.39 & 3.71\\
43 & sign or start a petition & 4.45 & 1.18 & 1.03 & 1.03 & 1.74\\
44 & mock or insult politicians who oppose the cause & 2.13 & 2.31 & 1.58 & 2.02 & 3.84\\
45 & join or form a trade/labor union & 4.19 & 1.40 & 1.00 & 1.07 & 1.70\\

46 & donate to charities who support the cause & 4.44 & 1.03 & 1.00 & 1.03 & 1.47\\
47 & refuse to eat (hunger strike) & 2.12 & 2.26 & 1.38 & 3.32 & 3.74\\
48 & damage public property (e.g., goverment buildings) & 1.38 & 3.67 & 3.55 & 3.65 & 4.68\\
49 & spread rumours about politicians who oppose the cause & 1.41 & 3.14 & 1.72 & 2.76 & 4.31\\
50 & participate in a public meeting of representatives and elected officials & 4.27 & 1.23 & 1.07 & 1.07 & 1.80\\

51 & physically harm oneself (e.g., setting oneself on fire) & 1.20 & 3.68 & 3.76 & 3.95 & 4.83\\
52 & boycott an election by not voting or spoiling one's ballot & 2.64 & 2.24 & 1.19 & 1.90 & 3.45\\
53 & stand in an election & 3.98 & 1.27 & 1.02 & 1.15 & 1.98\\
54 & soil politicians who oppose the cause (e.g., throwing eggs at them) & 1.43 & 3.28 & 3.11 & 3.19 & 4.47\\
55 & attack politicians with the intention of harming them (e.g., punching them) & 1.22 & 3.65 & 3.81 & 3.84 & 4.81\\

56 & refuse service (e.g., in a restaurant or shop) to individuals who oppose the cause & 1.86 & 3.03 & 1.46 & 2.65 & 4.16\\
57 & blackmail individuals who oppose the cause & 1.28 & 3.33 & 2.56 & 3.61 & 4.72\\
58 & threaten politicians who oppose the cause with physical harm & 1.16 & 3.57 & 3.76 & 3.81 & 4.81\\
59 & attack individuals who oppose the cause with the intention of harming them (e.g., punching them) & 1.24 & 3.76 & 3.88 & 3.88 & 4.79\\
60 & damage commerical property (e.g., shop windows) & 1.12 & 3.85 & 3.79 & 3.79 & 4.91\\

61 & attack politicians with the intention of killing them (e.g., stabbing them) & 1.11 & 3.93 & 4.00 & 4.00 & 4.91\\
62 & bribe politicians, representatives, and other elected officials & 1.17 & 3.02 & 1.28 & 3.04 & 4.85\\
63 & voting for candidates/parties & 4.51 & 1.19 & 1.09 & 1.07 & 1.63\\
64 & donate to political parties to make them change their position on an issue & 2.24 & 2.03 & 1.05 & 1.68 & 3.68\\
65 & blackmail politicians who oppose the cause & 1.31 & 3.15 & 2.15 & 3.54 & 4.64\\

66 & attack police officers and other government agents with the intention of killing them (e.g., stabbing them) & 1.07 & 4.00 & 4.00 & 4.00 & 4.98\\
67 & attack police officers and other government agents with the intention of harming them (e.g., punching them) & 1.11 & 3.81 & 3.95 & 3.95 & 4.95\\
68 & attack individuals who oppose the cause with the intention of killing them (e.g., stabbing them) & 1.06 & 3.94 & 4.00 & 4.00 & 4.92\\
69 & threaten individuals who oppose the cause with physical harm & 1.06 & 3.72 & 3.92 & 3.97 & 4.92\\
70 & attack members of the public (e.g., by setting off a bomb in a public place) & 1.05 & 3.90 & 3.92 & 3.92 & 4.92\\

71 & spread misinformation to influence public opinion & 1.24 & 3.41 & 1.41 & 2.95 & 4.76\\
72 & threaten to attack members of the public (e.g., by making a bomb threat) & 1.07 & 3.83 & 3.83 & 3.85 & 4.95

\end{xltabular}

\newpage

\setcounter{table}{2}

\begin{table}[t!]
\caption{Correlations between ratings in Study 2}
\centering
\begin{tabular}{rlrrrrr}
\toprule
\# & Rating & Acceptable & Disruptive & Violent & Extreme & Negative\\
\midrule1 & Acceptable &  & -.69 & -.64 & -.77 & -.89\\
2 & Disruptive & -.69 &  & .66 & .78 & .70\\
3 & Violent & -.64 & .66 &  & .79 & .64\\
4 & Extreme & -.77 & .78 & .79 &  & .77\\
5 & Negative & -.89 & .70 & .64 & .77 & \\
\bottomrule
\addlinespace
\end{tabular}
\caption*{\textit{Note.} Acceptable (1 = \textit{never}, 5 = \textit{always}); Disruptive, Violent, Extreme (1 = \textit{not at all}, 4 = \textit{very}); Negative (1 = \textit{very positive}, 5 = \textit{very negative})}
\end{table}

\end{document}